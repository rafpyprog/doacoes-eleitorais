%% abtex2-modelo-projeto-pesquisa.tex, v-1.9.6 laurocesar
%% Copyright 2012-2016 by abnTeX2 group at http://www.abntex.net.br/
%%
%% This work may be distributed and/or modified under the
%% conditions of the LaTeX Project Public License, either version 1.3
%% of this license or (at your option) any later version.
%% The latest version of this license is in
%%   http://www.latex-project.org/lppl.txt
%% and version 1.3 or later is part of all distributions of LaTeX
%% version 2005/12/01 or later.
%%
%% This work has the LPPL maintenance status `maintained'.
%%
%% The Current Maintainer of this work is the abnTeX2 team, led
%% by Lauro César Araujo. Further information are available on
%% http://www.abntex.net.br/
%%
%% This work consists of the files abntex2-modelo-projeto-pesquisa.tex
%% and abntex2-modelo-references.bib
%%

% ------------------------------------------------------------------------
% ------------------------------------------------------------------------
% abnTeX2: Modelo de Projeto de pesquisa em conformidade com
% ABNT NBR 15287:2011 Informação e documentação - Projeto de pesquisa -
% Apresentação
% ------------------------------------------------------------------------
% ------------------------------------------------------------------------

\documentclass[
	% -- opções da classe memoir --
	12pt,				% tamanho da fonte
	openright,			% capítulos começam em pág ímpar (insere página vazia caso preciso)
	oneside,			% para impressão em recto e verso. Oposto a oneside
	a4paper,			% tamanho do papel.
	% -- opções da classe abntex2 --
	%chapter=TITLE,		% títulos de capítulos convertidos em letras maiúsculas
	%section=TITLE,		% títulos de seções convertidos em letras maiúsculas
	%subsection=TITLE,	% títulos de subseções convertidos em letras maiúsculas
	%subsubsection=TITLE,% títulos de subsubseções convertidos em letras maiúsculas
	% -- opções do pacote babel --
	english,			% idioma adicional para hifenização
	french,				% idioma adicional para hifenização
	spanish,			% idioma adicional para hifenização
	brazil,				% o último idioma é o principal do documento
	]{abntex2}



% ---
% PACOTES
% ---

% ---
% Pacotes fundamentais
% ---

\usepackage[scaled]{uarial}
\renewcommand*\familydefault{\sfdefault} 
%\usepackage{lmodern}			% Usa a fonte Latin Modern
\usepackage[T1]{fontenc}		% Selecao de codigos de fonte.
\usepackage[utf8]{inputenc}		% Codificacao do documento (conversão automática dos acentos)
\usepackage{indentfirst}		% Indenta o primeiro parágrafo de cada seção.
\usepackage{color}				% Controle das cores
\usepackage{graphicx}			% Inclusão de gráficos
\usepackage{microtype} 			% para melhorias de justificação
% ---

% ---
% Pacotes de citações
% ---
\usepackage[brazilian,hyperpageref]{backref}	 % Paginas com as citações na bibl
\usepackage[alf]{abntex2cite}	% Citações padrão ABNT

% ---
% CONFIGURAÇÕES DE PACOTES
% ---

% ---
% Configurações do pacote backref
% Usado sem a opção hyperpageref de backref
\renewcommand{\backrefpagesname}{Citado na(s) página(s):~}
% Texto padrão antes do número das páginas
\renewcommand{\backref}{}
% Define os textos da citação
\renewcommand*{\backrefalt}[4]{
	\ifcase #1 %
		Nenhuma citação no texto.%
	\or
		Citado na página #2.%
	\else
		Citado #1 vezes nas páginas #2.%
	\fi}%
% ---

% ---
% Informações de dados para CAPA e FOLHA DE ROSTO
% ---
\titulo{Análise do Financimento de Campanhas Eleitorais Utilizando Ferramentas de Processamento de Linguagem Natural: Um Estudo de Caso das Eleiçoes para Prefeito no Município de São Paulo em 2012 e 2016.}
\autor{Rafael Alves Ribeiro}
\local{Brasil}
\data{2017}
\instituicao{%
   Instituto de Gestão e Tecnologia da Informação - IGTI
  }
\tipotrabalho{Tese (Doutorado)}
% O preambulo deve conter o tipo do trabalho, o objetivo,
% o nome da instituição e a área de concentração
\preambulo{Projeto de Trabalho de Conclusão de Curso apresentado ao
Instituto de Gestão e Tecnologia da Informação como requisito para
a conclusão do Curso de Pós-Graduação em Análise de Inteligência de Negócio.}
% ---

% ---
% Configurações de aparência do PDF final

% alterando o aspecto da cor azul
\definecolor{blue}{RGB}{41,5,195}

% informações do PDF
\makeatletter
\hypersetup{
     	%pagebackref=true,
		pdftitle={\@title},
		pdfauthor={\@author},
    	pdfsubject={\imprimirpreambulo},
	    pdfcreator={LaTeX with abnTeX2},
		pdfkeywords={abnt}{latex}{abntex}{abntex2}{projeto de pesquisa},
		colorlinks=true,       		% false: boxed links; true: colored links
    	linkcolor=black,          	% color of internal links
    	citecolor=black,        		% color of links to bibliography
    	filecolor=magenta,      		% color of file links
		urlcolor=blue,
		bookmarksdepth=4
}
\makeatother
% ---

% ---
% Espaçamentos entre linhas e parágrafos
% ---

% O tamanho do parágrafo é dado por:
\setlength{\parindent}{1.3cm}

% Controle do espaçamento entre um parágrafo e outro:
\setlength{\parskip}{0.2cm}  % tente também \onelineskip

% ---
% compila o indice
% ---
\makeindex
% ---

% ----
% Início do documento
% ----
\begin{document}


% Seleciona o idioma do documento (conforme pacotes do babel)
%\selectlanguage{english}
\selectlanguage{brazil}

% Retira espaço extra obsoleto entre as frases.
\frenchspacing

% ----------------------------------------------------------
% ELEMENTOS PRÉ-TEXTUAIS
% ----------------------------------------------------------
% \pretextual

% ---
% Capa
% ---
\imprimircapa
% ---

% ---
% Folha de rosto
% ---
\imprimirfolhaderosto
% ---

% ---
% NOTA DA ABNT NBR 15287:2011, p. 4:
%  ``Se exigido pela entidade, apresentar os dados curriculares do autor em
%     folha ou página distinta após a folha de rosto.''
% ---

% ---
% inserir lista de ilustrações
% ---
%\pdfbookmark[0]{\listfigurename}{lof}
%\listoffigures*
%\cleardoublepage
% ---

% ---
% inserir lista de tabelas
% ---
%\pdfbookmark[0]{\listtablename}{lot}
%\listoftables*
%\cleardoublepage
% ---

% ---
% inserir lista de abreviaturas e siglas
% ---
%\begin{siglas}
%  \item[ABNT] Associação Brasileira de Normas Técnicas
%  \item[abnTeX] ABsurdas Normas para TeX
%\end{siglas}
% ---

% ---
% inserir lista de símbolos
% ---
%\begin{simbolos}
%  \item[$ \Gamma $] Letra grega Gama
%  \item[$ \Lambda $] Lambda
%  \item[$ \zeta $] Letra grega minúscula zeta
%  \item[$ \in $] Pertence
%\end{simbolos}
% ---

% ---
% inserir o sumario
% ---
\pdfbookmark[0]{\contentsname}{toc}
\tableofcontents*
\cleardoublepage
% ---


% ----------------------------------------------------------
% ELEMENTOS TEXTUAIS
% ----------------------------------------------------------
\textual

% ----------------------------------------------------------
% Problema de Pesquisa
% ----------------------------------------------------------
\chapter{Problema de Pesquisa}

\index{elementos textuais}“O financiamento de campanhas eleitorais é um
componente de grande importância no funcionamento de regimes democráticos” \cite{sztutman2013financiamento}. No Brasil, historicamente tem prevalecido o
financiamento privado de campanhas por pessoas jurídicas, agentes com grande
poder econômico e forte capacidade de organização.

Apesar de serem um instrumento legal de participação política, as doações
privadas realizadas por grupos empresariais podem ser consideradas um fator
de desequilíbrio no jogo democrático. Tanto \citeonline{claessens2008political},
como \citeonline{boas2014spoils} ressaltam que as conexões politicas criadas pelas empresas com doações a campanhas eleitorais resultam em algum tipo de favorecimento político. Este favorecimento se traduz em uma maior expectativa de retorno dos investidores, no acesso preferencial as fontes de financiamento e em um maior número de contratos públicos atendidos.

Evidenciando ainda mais o efeito danoso da atuação de grupos empresariais no financiamento de campanhas, em 2014, a segunda etapa da operação Lava Jato trouxe à tona uma série de escândalos de prática de lavagem de dinheiro, corrupção e fraudes em licitações envolvendo grandes grupos empresariais organizados em cartel e a estatal Petrobras.

Deste cenário, surge a mola propulsora que motiva o desenvolvimento deste trabalho:

Considerando-se os pleitos de 2012 e 2016, qual o efeito da proibição do financiamento político realizado por pessoas jurídicas nas receitas e despesas das campanhas para o cargo de prefeito do município de São Paulo?


% ----------------------------------------------------------
% Objetivo
% ----------------------------------------------------------
\chapter[Objetivo]{Objetivo}

Para identificar o efeito da proibição do financiamento político realizado por pessoas jurídicas, o presente trabalho propõe-se a classificar e comparar os gastos  realizados nas campanhas eleitorais para o cargo de prefeito nas eleições municipais de São Paulo nos pleitos de 2012 e 2016. Para tal, pretende-se:

\begin{alineas}
 \item Realizar o levantamento e tratamento dos dados do Repositório de dados eleitorais do Tribunal Superior Eleitoral;
 \item Criar um modelo de classificação dos tipos de receitas e despesas realizadas pelos candidados;
 \item Descrever e analisar os dados e os resultados obtidos com o modelo de classificação;
 \item Apresentar o resultado obtido em forma de artigo científico.
\end{alineas}


% ----------------------------------------------------------
% Justificativa
% ----------------------------------------------------------
\chapter[Justificativa]{Justificativa}

Nos últimos anos o sistema de financiamento eleitoral brasileiro passou por transformações que alteraram significativamente as fontes de recursos disponíveis para as campanhas políticas. Com a análise dos dados contidos no repositório de dados de dados eleitorais, este trabalho se propõe a promover uma melhor compreensão da dinâmica eleitoral do país, extraindo informações que possam servir como referência para futuros estudos e discussões sobre o tema.

Com objetivo de garantir autenticidade do processo eleitoral e assegurar  transparência das relações entre o Poder Público e os agentes econômicos, em 2011, o Supremo Tribunal Federal deu início a Ação Direta de Inconstitucionalidade 4.650. O julgamento da ação foi concluído em setembro de 2015 com a declaração da inconstitucionalidade dos dispositivos legais que autorizavam as contribuições de pessoas jurídicas às campanhas eleitorais.

A ação do Supremo Tribunal e a pressão dos setores da sociedade culminaram na  realização da Reforma Eleitoral 2015 (Lei nº 13.165/2015), que proibiu o financiamento eleitoral por pessoas jurídicas, restringindo o aporte de recursos a doações de pessoas físicas e valores do Fundo Partidário.

Com base neste cenário, este trabalho se propõe a realizar uma contribuição para a compreensão dos efeitos das mudanças das regras para o financiamento eleitoral,  lançando luz sobre as alterações na dinâmica da utilização de recursos das campanhas após a proibição da doação de recursos por pessoas jurídicas.

Adicionalmente, espera-se que o estudo possa contribuir para a implementação de mecanismos mais eficazes na detecção de indícios de irregularidades na prestação de contas de campanha no que tange ao limite de gastos, arrecadação e aplicação de recursos.


% ----------------------------------------------------------
% Metodologia
% ----------------------------------------------------------
\chapter[Metodologia]{Metodologia}


Segundo \apudonline[p.~155]{ander1978introduccion}{lakatos2001metodologia}, a pesquisa é um “procedimento reflexivo sistemático, controlado e crítico, que permite descobrir novos fatos ou dados, relações ou leis, em qualquer campo do conhecimento”.

De acordo com \citeonline{cervo2007roberto}, a pesquisa parte, pois, de uma dúvida ou problema e, com o uso do método científico, busca uma resposta ou solução.

Entende-se que utilização da pesquisa exploratória, se justifica pela necessidade. De utilização de métodos e técnicas para a análise dos dados do repositorio eleitoral com vistas oferecer informações sobre as alteraçãoes na dinâmica dos gastos em campanhas eleitorais. \cite{cervo2007roberto}

Utilizando-se de uma abordagem quantitativa, o trabalho pretende abordar o problema através de um estudo de caso de natureza aplicada para realizar o estudo dos dados divulgados pelo Tribunal Superior Eleitoral no Repositório de dados eleitorais:

\begin{citacao}
 O Repositório de dados eleitorais é uma compilação de informações brutas das eleições, desde as de 1945, voltada para pesquisadores, imprensa e pessoas interessadas em analisar os dados de eleitorado, candidaturas, resultados e prestação de contas. \cite{brasil2017repositorio}
\end{citacao}

Para a análise dos dados pretende-se desenvolver um classificador de texto que permita reduzir a dimensionalidade dos dados, com ênfase na análise léxica e na utilização modelo Bayesiano Ingênuo ou Simples de aprendizagem supervisionada para a para categorização dos dados.

% ---
% Finaliza a parte no bookmark do PDF
% para que se inicie o bookmark na raiz
% e adiciona espaço de parte no Sumário
% ---
\phantompart

\postextual

% ----------------------------------------------------------
% Referências bibliográficas
% ----------------------------------------------------------
\citeoption{ABNT-final}
\bibliography{projeto-pesquisa}


\phantompart
\end{document}
